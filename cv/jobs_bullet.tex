\cvsection{Work Experience}
\begin{cventries}
\cventry
	{Android \& Scala Software Engineer}
{Depop}
	{London, UK}
	{November 2017 - \present{}}
{%
\begin{cvitems} 
\item 
	Develop new features for the Depop marketplace app
	1,000,000+ installations worldwide, 4.2 stars rating. 
	 Key functionalities I have implemented: Instagram-like 60fps gallery image picker, mark item as shipped, share profile shop, and more. 
\item Scala development of a costellation of mobile-specific api microservices using play; codebase is heavily shaped on FP patterns (applicative, monads), as approach I follow contract-driven development (tools: apiary, blueprint and dredd). Infrastructure runs on AWS, for monitoring librato is used. 
\item Work hard to modernise the android codebase and build re-usable library components for the teams 
\item Driven the adoption of Kotlin as main language to develop every new feature, in particular emphasizing functional capabilities of the language and spreading knowledge about coroutines-based concurrency patterns 
\item Advocating adoption of feature modules to restructure the monolithic structure of the android codebase, in order to make tdd effective, increase decoupling, ultimately allowing the company to scale the structure of the teams 
\item Pairing, coaching and mentoring. My approach encourages a critical attitude to reason about the problem, understand the (technical and non-technical) constraints,  question about the possible strategies and their tradeoffs, and evaluate pros and cons of different solutions
 \item Responsible of defining a testing strategy to drive adoption of BDD and TDD across the whole mobile (android and ios) teams. Tools: apiary, dredd, wiremock, junit, mockito, espresso
\item Promoting communication between team members and between teams at all levels
\item We follow scrum, 2-weeks sprints with release at the end of each sprint (for mobile teams) and kanban and CI/CD (for backend)
\item App is freely available from 
	\url{https://play.google.com/store/apps/details?id=com.depop&hl=en_GB} 
	\end{cvitems}
}

\cventry
	{Software engineer}
{Sainsbury's}
	{London, UK}
	{February 2016 - October 2017}
{%
\begin{cvitems} 
\item 
	Leading the android development of the Grocery 
   native app for shopping online. 
	150k+ users UK-wise, 4 stars rating on the playstore even though the app consumes data from a quite unreliable source (plain html scraping)
\item 
	Implement, maintain and deploy  
		new features at consistent speed (bi-weely release cycle), without compromising code quality and with proven ability to arrange urgent tasks and contingencies.
	 Key functionalities I have implemented: security enhancements, 3d-secure payments in app, push notifications, and more
	\item Be proactive, take ownership and responsibility on critical choices to enhance the value of the product
	\item Find the right tradeoff between 
		short workarounds vs long-term solutions to meet business and customer needs 
	\item Refactor codebase, promoting adoption of TDD/BDD, clean, MVP, Rx, Dagger, retrofit and functional patterns
    \item Be a go-to person, do mentorship and training
%\item 
%I'm used to prepare slides to explain benefits and efforts of some implementation choices also to non-technical colleagues
    %\item I am assistant coordinator of the functional programming ``guild'', an internal bi-weekly meeting aimed at promoting the adoption of FP within Sainsbury's
    \item 
    I have been praised by the company for my hard work
\item \textbf{Technologies}: 
    RxJava2, Dagger2, Retrofit2 (I have implemented custom converter to hide the scraping part), Jsoup,
    Realm,
   Fresco, firebase, appsflyer. Tests: junit4, mockito, espresso, appium, wiremock.
	\item  CI/CD: I setup the delivery pipeline running
		unit tests, espresso tests on aws farm, jacoco, sonarqube, lint
\item App is freely available from 
	\url{https://play.google.com/store/apps/details?id=com.sainsburys.gol&hl=en}
	\end{cvitems}
}
\cventry
{Android software engineer}
{Spotlime}
{Milan, Italy}
{September 2014 -- December 2015 }
{
	\begin{cvitems}
	\item Leading the 
   development of the 
   Android fully native application for 
   Spotlime, a startup aimed  at promoting  the 
   discovery of the best events in Milan, Rome, etc. 6+k users,  4+ stars on the playstore. 
      \item Implement \emph{from scratch} and mantain the Spotlime android
	      native application (i was the only android dev)
      \item 
   Work alongside the marketing team to deliver a user-centric product closer
   to real customer 
   needs, promoting solutions that helps improving UX in response
   to users' feedback and analytics data.
      \item 
   Work alongside server-side and iOS developers to coordinate the upcoming
   releases, following CTO directives.
\item Help to develop Spotlime booking desktop website using mainly server-side JSF and
   primefaces.
      \item Self-managed, self motivated, able to work with little or no supervision
\item Key functionalities: 
	client-server data synchronization, offline first app managed from scratch, 
   advanced geolocation and geofencying, Facebook login, push notifications, in-app cc payments.
\item \textbf{Technologies}: sqlite3, Ormlite, Glide, GA and appsflyer
   \item App is freely available from 
   \url{https://play.google.com/store/apps/details?id=com.gooutsrl&hl=en}
	\end{cvitems}
}

\cventry
{Lead Quality Assurance Analyst} 
{Usablenet - leading global mobile and multi-channel technology company.}
   {Udine, Italy}
{October 2012 -- August 2014}
   {
	\begin{cvitems}
	\item After few months working as a quality assurance analyst, I was
      promoted 
   quality assurance team leader at Usablenet, a leading 
  technology platform  company delivering enterprise-level mobile and
  multichannel commerce solutions. Team of 7 QA people, for a total number of 200 projects. 
\item 
  Working in synergy with mobile web
   developers, project manager, solution engineer team and customers to
   consistently deliver high-quality
   products that fulfills customer expectations and end user needs. 
\item Monitor and track project status during all steps, to meet deadlines
   and ensure all scope changes,
 variances and contingences that may arise during the 
 projects lifecycle were visible to all people involved.
 \item 
   Ability to prioritize and track multiple projects in parallel,
    manage  allocation of resources within the team, supervise and support my team
    activities.
 \item  Be a go-to person.
 \item 
   Proven ability to work under pressure responsibly and fulfilling high
   expectations.
%\item test plan and test cases creation using TestLodge
\item Promote new strategies to speed up effective communication among all
   teams involved in the project, suggesting 
   improvements to current workflows and defining new internal procedures.
\item Debugging and monitoring of %customized web analytics solutions and 
technical requirements (for example requirements involving
   akamai technologies).
\item 247 QA support.
   \item Developing small scripts (mainly using Python+JQL) which help to retrieve project
 information from Atlassian Jira and provide automatic statistical reports about
 project status.
   \item Personally responsible as QA analyst of:
   Camelot Group (UK National Lottery, including the
   launch of the mobile website for the new lotto raffle on October 2013), Dell Inc. (22 countries
   including US, UK, China etc), FedEx,
   MaryKay, Selfridges, Surfstitch and many others.
%which requires ability to quickly gain
%familiarity with unknow projects, providing insights which help to spot the root
%cause of the issues and testing if it has
%been solved successfully.
\item Company website: \url{http://usablenet.com/}
	\end{cvitems}
}
\cventry
{Teacher (on call)}
{University of Udine}
{Udine}
{2006 -- 2012}
{
	\begin{cvitems}
	\item 
   I have been asked for six years to teach  the \emph{Esercitazioni guidate di Fisica per il Corso
      di preparazione al Test di ammissione alla Facolt\`a di Medicina e
      Chirurgia} at the University of Udine (support training lectures
   organized
for candidates to the placement exam for accessing the courses at the faculty
of Medicine).
	\end{cvitems}
}
\cventry
{IT technical expert (on call)}
{R.U.E. Risorse Umane Europa (no-profit
   association)}
{Udine}
{September~2011 -- August~2012}
{
	\begin{cvitems}
	\item 
   Handling the IT issues in a small office (5 employees): pc, network and
   website maintenance
	\end{cvitems}
   }
\cventry
{\LaTeXe\ typesetter}
{Consorzio per la Fisica}
{Trieste}
{January 2010 -- December 2010}
{   
	\begin{cvitems}
	\item 
I typeset prof. E.~Gozzi lectures notes for his course of Quantum Mechanics using custom \LaTeXe\ style and including the images/plots
\item
   The notes are freely  available at 
   \url{http://www-dft.ts.infn.it/~gozzi/QM2.pdf}
	\end{cvitems}
		}
\cventry
{C++ developer as volounteer}
{I.N.F.N. (Istituto Nazionale di Fisica
   Nucleare)}
{Trieste}
{February~2008 -- May~2008}
{Development of a object-oriented C++ library for uniform and non-uniform
   pseudo-random number generations, including some cutting edge algorithms}

\end{cventries}
