\cvsection{Education}

\begin{cventries}
\cventry
{Master student in Theoretical Physics}
{University of Trieste}
{Trieste (Italy)}
{September 2009 -- now}%
{%
      \begin{cvitems}
      \item{%
   Still  under completion (due to work). I have successfully done  exams including: advanced statistical mechanics,
   quantum field theory, advanced mathematical methods and computational physics,
   including C++
   implementation and comparison of symplectic algorithms for numerical
   integration of ordinary differential initial valued problems in classical
   molecular dynamics and monte carlo simulation of statistical mechanics
   system like the spin Ising model
}
\end{cvitems}
}
%\end{shaded}


\cventry
{Bachelor's degree in Physics 110/110}
{University of Trieste}
{Trieste (Italy)}
{May~2009}
{%
      \begin{cvitems}
      \item{%
   Thesis title: \emph{Simulazione numerica dello stress termomeccanico in un
      ellissometro}.
}
\item{%
   Advisor: Prof. E.~Milotti.
}
\item{%
   Description:
numerical simulation of the influence of laser beam-pointing fluctuations 
on the
   thermomechanical stress-induced birefringence in the optical
   ellipsometer of PVLAS experiment. The thesis required a deep understanding
   of the physics involved, knowledge of stochastic processes, familiarity with signal processing and 1~year of advanced C++ programming
   including the development of an object-oriented library for finite-element
   solution of partial differential equations 
   in rectangular domains cooupled to numerical integration of Ito stochastic differential
   equations.
Thesis available (in Italian) at 
\url{http://www.infn.it/thesis/PDF/getfile.php?filename=3304-Candolini-triennale.pdf}
}
\end{cvitems}
}
    

\cventry
{Scientific high school diploma 100/100} 
{Liceo Scientifico Statale G. Marinelli (Scientific diploma, high school)}
{Udine (Italy)}
{July~2005}
{%
      \begin{cvitems}
      \item{%
   High school's thesis in physics: \emph{Approccio spazio-temporale globale alla
      teoria quantistica e formulazione di Feynman della QED} (translation:
   overall space-time approach to quantum theory and Feynman's formulation of
   QED). Advisor: Prof.
   F.~de~Stefano
}
\end{cvitems}
}
 


\end{cventries}
