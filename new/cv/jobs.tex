%  \cventry
%    {Software Engineer \& Security Researcher (Compulsory Military Service)}
%    {R.O.K Cyber Command, MND}
%    {Seoul, S.Korea}
%    {Aug. 2014 - Exp. Apr. 2016}
%    {
%      \begin{cvitems}
%        \item {Implemented a military cooperation system which is web based real time messenger in Scala on Lift.}
%        \item {Improved functionality on military command and control system for incident response with Java Servlet.}
%        \item {Lead engineer on agent-less backtracking system that can discover client device's fingerprint(including public and private IP) independently of the Proxy, VPN and NAT.}
%      \end{cvitems}
%    }




\cvsection{Employment}
\begin{cventries}
\cventry
{software development}
{Sainsburys - Digital Lab}
{London, UK}
{February 2016 -- present}
{%
   \textbf{Overview:}
   I spent the first couple  of months introducing TDD practises in the
   development cycle of the 
   Grocery App app, 
   spinning up and maintaining a Jenkins 2.x  CI/CD
   OSX server for android and iOS, and writing multi-platform QA-compliant
   E2E acceptance appium Java tests (with page object pattern),
    providing training to promote the adoption of
    these practises across company's projects and teams.
   %Then, I have been asked to build as Android library a 
%collection of tools aimed at simplifying the development of Android apps, with
%primary emphasis on analytics. This library 
%has been developed
%following TDD approach, code coverage is close to 100\% and
%now it is now a component 
%inside several company's apps. 
   Since June, I've been leading the android development of the Grocery 
   native app, adding new features including e-voucher wallet and nectar point
   redemption, advanced filters, weighted items and more.  Under the hood the
   app poses intriguing technical challenges because although the app itself is
   purely native, the data are provided via real-time dynamic scraping  in-app
   of the Sainsbury's grocery website, for which a fine-tuned combination of
   jsoup and android webview for js execution when appropriate is used to boost
   speed and to narrow the impact of scraping on the performance.  The project
   lifecycle follows an agile approach with two-weeks sprints, sprint planning,
   retro and sprint demo,  and with one
    release on the playstore at the end of each sprint.  Currently, the app has
    more than 50k unique users.
    \newline
    I have been praised by the company for my hard work. 
   \newline
    \textbf{Preferred technology stack:} Jsoup, Retrofit2, Dagger2, RxJava,
    Realm,
   Fresco. For testing: junit4, testng, mockito, espresso, appium, wiremock.
   Architectual patterns: clean architecture with mvp on presentation layer. 
   \newline The Android app is freely downloadable from Google play store:
   \url{https://play.google.com/store/apps/details?id=com.sainsburys.gol&hl=en}
}
\cventry
{Android software engineer}
{Spotlime}
{Milan, Italy}
{September 2014 -- December 2015 }
{
   \textbf{Overview:}
   Leading the 
   development of the 
   Android native application for 
   Spotlime, a startup aimed  at promoting  the 
   discovery of the best events in Milan, Rome and Florence. 
   \newline
   \textbf{Duties and responsibilities:}
   \begin{itemize}
      \item Development \emph{from scratch} and maintenance of the Spotlime android
	 native application.
      \item Provide level of effort to integrate new functionalities and
	implement. 
      \item 
   Work alongside the marketing team to deliver a user-centric product closer
   to real customer 
   needs, promoting solutions that helps improving UX in response
   to users' feedback and analytics data.
      \item 
   Work alongside server-side and iOS developers to coordinate the upcoming
   releases, following CTO directives.
\item Help to develop Spotlime booking desktop website using mainly server-side JSF and
   primefaces.
\end{itemize}
\textbf{Preferred technologies}: Realm database, Retrofit, Glide, EventBus,
Gson.
   Experience in: REST, client-server data synchronization, offline mode,
   advanced geolocation techniques, Facebook SDK, GCM push notifications
   \newline
   \textbf{More details:}
   %Company website:
   %\url{http://www.spotlimeapp.com/}
   The Android app is freely downloadable from Google play store: 
   \url{https://play.google.com/store/apps/details?id=com.gooutsrl&hl=en}
}

\cventry
{Lead Quality Assurance Analyst} 
{Usablenet - leading global mobile and multi-channel technology company.}
   {Udine, Italy}
{October 2012 -- August 2014}
   {
      After few months working as a quality assurance analyst, I was
      promoted 
   quality assurance team leader at Usablenet, a leading 
  technology platform  company delivering enterprise-level mobile and
  multichannel commerce solutions.
   My responsibilities:
   \begin{itemize}
\item 
  Working in synergy with mobile web
   developers, project manager, solution engineer team and customers to
   consistently deliver high-quality
   products that fulfills customer expectations and end user needs. 
\item Monitor and track project status during all steps, to meet deadlines
   and ensure all scope changes,
 variances and contingences that may arise during the 
 projects lifecycle were visible to all people involved.
 \item 
   Ability to prioritize and track multiple projects in parallel,
    manage  allocation of resources within the team, supervise and support my team
    activities.
 \item  Be a go-to person.
 \item 
   Proven ability to work under pressure responsibly and fulfilling high
   expectations.
%\item test plan and test cases creation using TestLodge
\item Promote new strategies to speed up effective communication among all
   teams involved in the project, suggesting 
   improvements to current workflows and defining new internal procedures.
\item Debugging and monitoring of %customized web analytics solutions and 
technical requirements (for example requirements involving
   akamai technologies).
\item 247 QA support.
   \item Developing small scripts (mainly using Python+JQL) which help to retrieve project
 information from Atlassian Jira and provide automatic statistical reports about
 project status.
 \end{itemize} 
   Personally responsible as QA analyst of:
   Camelot Group (UK National Lottery, including the
   launch of the mobile website for the new lotto raffle on October 2013), Dell Inc. (22 countries
   including US, UK, China etc), FedEx,
   MaryKay, Selfridges, Surfstitch and many others.
%which requires ability to quickly gain
%familiarity with unknow projects, providing insights which help to spot the root
%cause of the issues and testing if it has
%been solved successfully.
Company website: \url{http://usablenet.com/}}
\cventry
{Teacher (on call)}
{University of Udine}
{Udine}
{2006 -- 2012}
{
   I have been asked to teach  the \emph{Esercitazioni guidate di Fisica per il Corso
      di preparazione al Test di ammissione alla Facolt\`a di Medicina e
      Chirurgia} at the University of Udine (support training lectures
   organized
for candidates to the placement exam for accessing the courses at the faculty
of Medicine).
}
\cventry
{IT technical expert (on call)}
{R.U.E. Risorse Umane Europa (no-profit
   association)}
{Udine}
{September~2011 -- August~2012}
{
   Handling the IT issues in a small office (5 employees): pc, network and
   website maintenance}
\cventry
{\LaTeXe\ typesetter}
{Consorzio per la Fisica}
{Trieste}
{January 2010 -- December 2010}
{   I typeset prof. E.~Gozzi lectures notes for his course of Quantum Mechanics.
   The notes are currently available at 
   \url{http://www-dft.ts.infn.it/~gozzi/QM2.pdf}}
\cventry
{C++ developer as volounteer}
{I.N.F.N. (Istituto Nazionale di Fisica
   Nucleare)}
{Trieste}
{February~2008 -- May~2008}
{Development of a object-oriented C++ library for uniform and non-uniform
   pseudo-random number generations, including some cutting edge algorithms}

\end{cventries}
