
\documentclass[10pt,a4paper]{moderncv}   % possible options include font size ('10pt', '11pt' and '12pt'), paper size ('a4paper', 'letterpaper', 'a5paper', 'legalpaper', 'executivepaper' and 'landscape') and font family ('sans' and 'roman')


% moderncv themes
\moderncvstyle{casual}                        
%\moderncvstyle{classic}                        
%\moderncvstyle{oldstyle}                        
%\moderncvstyle{banking}    
% style options are 'casual' (default), 'classic', 'oldstyle' and 'banking'
\moderncvcolor{blue}                          
% color 'blue' (default), 'orange', 'green', 'red', 'purple', 'grey' and 'black'

%\renewcommand{\familydefault}{\sfdefault}    % to set the default font; use '\sfdefault' for the default sans serif font, '\rmdefault' for the default roman one, or any tex font name

%\nopagenumbers{}                             % uncomment to suppress automatic page numbering for CVs longer than one page

% character encoding
%\usepackage[utf8]{inputenc}                  % if you are not using xelatex ou lualatex, replace by the encoding you are using

% adjust the page margins
\usepackage[scale=0.75]{geometry}
\setlength{\hintscolumnwidth}{3cm}           % if you want to change the width of the column with the dates
%\setlength{\makecvtitlenamewidth}{10cm}      % for the 'classic' style, if you want to force the width allocated to your name and avoid line breaks. be careful though, the length is normally calculated to avoid any overlap with your personal info; use this at your own typographical risks...

\usepackage[english]{babel}

\usepackage{framed}
\definecolor{shadecolor}{gray}{.80}%

%\usepackage{fontspec}
\usepackage{microtype}
%\defaultfontfeatures{Ligatures=TeX, Scale=MatchLowercase}


%\setmainfont[SmallCapsFeatures={LetterSpace=6}, Numbers={Proportional,OldStyle}]{Minion Pro}

%%
%\setmainfont[SizeFeatures = {
%  {Size =     -8.41, OpticalSize = 8},
%  {Size = 8.41-13.1, OpticalSize = 11},
%  {Size = 13.1-20.0, OpticalSize = 19},
%  {Size = 20.0-,     OpticalSize = 72}}
%]{Minion Pro}

%\setmainfont[
%SizeFeatures = {
%  {Size =     -8.41, OpticalSize = 8},
%  {Size = 8.41-13.1, OpticalSize = 11},
%  {Size = 13.1-20.0, OpticalSize = 19},
%  {Size = 20.0-,     OpticalSize = 72}},
%%Ligatures          = {Required, Common, Contextual, TeX},
%%Numbers            = {OldStyle, Proportional},
%%RawFeature         = {expansion  = default},
%%UprightFeatures    = {RawFeature = {protrusion = mnr}},
%%BoldFeatures       = {RawFeature = {protrusion = mnrb}},
%%ItalicFeatures     = {RawFeature = {protrusion = mni}},
%%BoldItalicFeatures = {RawFeature = {protrusion = mnib}}
%]{Minion Pro}


%\setsansfont[LetterSpace=3, Numbers={Proportional,OldStyle}]{Myriad Pro}

\usepackage{array}
\newcolumntype{L}[1]{>{\raggedright\let\newline\\\arraybackslash\hspace{0pt}}m{#1}}
\newcolumntype{C}[1]{>{\centering\let\newline\\\arraybackslash\hspace{0pt}}m{#1}}
\newcolumntype{R}[1]{>{\raggedleft\let\newline\\\arraybackslash\hspace{0pt}}m{#1}}

\newenvironment{mycvline}{
      \begin{tabular*}{\textwidth}{ R{\hintscolumnwidth}
	    @{\extracolsep{\columnsep}}l}
   }{
\end{tabular*}
}

%\SetProtrusion
%   [ name     = min-eu2 ]
%   { encoding = {EU2},
%     family   = MinionPro  }
%   {
%    {,} = {  ,500},
%     -  = {  ,500}
%   }
%

%\usepackage[fullfamily,opticals,mathlf,onlytext]{MinionPro}



\newcommand{\mail}[1]{\href{mailto:#1}{\texttt{#1}}}
%\newcommand{\website}[1]{\href{#1}{\texttt{#1}}}
\newcommand{\website}[2]{\href{#1#2}{\texttt{#2}}}
\newcommand{\skype}[1]{\href{skype:#1?userinfo}{#1}}%
   % http://stackoverflow.com/questions/16177332/create-application-link-to-skype-profile
% personal data
\firstname{Alessandro}
\familyname{Candolini}
\title{%
%      \fontspec[Variant = 2]{Zapfino} Curriculum
 %     {Vit\fontspec[Variant = 3]{Zapfino}\ae}
 Curriculum Vit\ae
}
\address{via Monte Grappa~44}{33100 Udine (Italy)}    
\mobile{(+39)~333~4387420}
\phone{Skype: alessandro.candolini}
%\fax{+3~(456)~789~012}                        
\email{alessandro.candolini@gmail.com}
%\homepage{www.johndoe.com}                    
%\extrainfo{additional information}            
%\extrainfo{%
%   \begin{tabular}{c}
%   Linkedin:  \website{https://}{www.linkedin.com/in/alessandrocandolini} \\
%   Bitbucket:  \website{https://}{bitbucket.org/acando86}\\
%   Stackoverflow: 
%   \website{http://}{stackoverflow.com/users/5477611/alessandro-candolini} \\
%   Kaggle profile:  \website{http://}{www.kaggle.com/alessandrocandolini}
%\end{tabular}
%}
%\photo[64pt][0pt]{profile2}
\photo[64pt][0pt]{foto2}
% '64pt' is the height the picture must be resized to, 0.4pt is the thickness of the frame around it (put it to 0pt for no frame) and 'picture' is the name of the picture file; optional, remove the line if not wanted
%\quote{Some quote (optional)}                 % optional, remove the line if not wanted





% to show numerical labels in the bibliography (default is to show no labels); only useful if you make citations in your resume
%\makeatletter
%\renewcommand*{\bibliographyitemlabel}{\@biblabel{\arabic{enumiv}}}
%\makeatother

% bibliography with mutiple entries
%\usepackage{multibib}
%\newcites{book,misc}{{Books},{Others}}
%----------------------------------------------------------------------------------
%            content
%----------------------------------------------------------------------------------
\begin{document}
%\begin{CJK*}{UTF8}{gbsn}                     % to typeset your resume in Chinese using CJK
%-----       resume       ---------------------------------------------------------
\makecvtitle

   
%\section{Bachelor thesis}
%\cvitem{title}{\emph{Simulazione numerica dello stress termomeccanico in un
%      ellissometro}}
%\cvitem{supervisors}{Prof. E.~Milotti}
%\cvitem{description}{

%\begin{tabular*}{7in}{l@{\extracolsep{\fill}}l}

I am an Italian theoretical particle physicist with heavy mathematical
background, problem-solving attitude, and passion for statistical data
analysis, simulation and software development.
My current interest is in data science and I am looking for a position sitting at the crossroad
between 
modern machine learning algorithms, traditional statistics and the computational
challenges of writing code to process effectively massive datasets. 
I firmly believe in the future of this exiting and cross-disciplinary field and at the same time I think it is
close to my academic background  and fits my skills. 
My ideal job would involve working in a stimulating, innovative,
environment where I can give
valuable insights that help the 
company to grow  and at the same time where I can  
improve myself and my knowledge.


\section{Contact information}

\begin{mycvline}
     Home address: & via Monte Grappa 44, 33100, Udine, Italy  \\
      E-mail: & \mail{alessandro.candolini@gmail.com} \\
    Mobile phone: & 
    \href{tel:00393334387420}{(+39)~333~4387420}\\
    Skype: & \skype{alessandro.candolini} \\
    Linkedin: & \website{https://}{www.linkedin.com/in/alessandrocandolini}  \\
    Bitbucket:  & \website{https://}{bitbucket.org/acando86} \\
    Stackoverflow:  & 
   \website{http://}{stackoverflow.com/users/5477611/alessandro-candolini} \\
    Kaggle profile:  & \website{http://}{www.kaggle.com/alessandrocandolini}
\end{mycvline}



\section{Work experiences}
\cventry{September 2014 -- present}{Spotlime}{Mobile software engineer}{Milan, Italy}{}{
   I am responsible of the implementation design, development, QA and
   release of 
   native Android application for Spotlime, a startup that
   provides an app to discover the best events in Milan and Rome. 
   \textbf{My duties and responsibilities}:
   \begin{itemize}
      \item develop from scratch the production-level Android native application  for Spotlime;
      \item provide level of effort to support new functionalities and
	 implement them, 
      choosing the best implementation strategy, reviewing suitable libreries,
      and evaluating the best way to
	 achieve the expected behavior;
      \item test and bug fixing;
      \item 
   work alongside marketing team to deliver a product closer
   to real user
   needs, promoting solutions that helps improving user experience in response
   to end user feedback and analytics data. 
      \item 
   work alongside server-side and iOS developers to coordinate the upcoming
   releases.
   \end{itemize}
   %Functionality includes: Fragments to deliver a fluent user experience,
   %database, geolocation service and maps, FB login, paypal payment, analytics.
   \textbf{Techonology used:}
   Android SDK, 
   Android Studio IDE, Eclipse IDE, Facebook SDK, Google Play Services SDK,
   ApplsFlyer, Support Library, ORMLite, sqlite, GSon,  eventbus.
   Experience in: client-server synchronization,  multithreading, 
   credit card payment, integration with social channels,
   analytics, maps and geolocation. 
   Company website:
   \url{http://www.spotlimeapp.com/it/home/}.
   The app is freely available from Google play store: 
   \url{https://play.google.com/store/apps/details?id=com.gooutsrl&hl=en}
}
%\cventry{July 2015 -- present}{Freelance}{Part-time web developer}{Udine,
%   Italy}{}{
%   Working on small projects to build simple websites using mainly Wordpress.
%   Projects include: grupoeuroamerica.
%   \url{http://www.grupoeuroamerica.com.mx}
%}
\cventry{October 2012 -- August 2014}{Usablenet - leading global mobile and
   multi-channel technology company.}{Lead Quality Assurance
   Analyst} 
   {Udine, Italy}{}{After few months working as a quality assurance analyst, I
      have soon become
   quality assurance team leader at Usablenet, a multichannel technology
   company. 
   \\
   \textbf{My duties and responsibilities}:
   \begin{itemize}
\item 
  working in synergy with  web
   developers, project manager, solution engineer team and client to
   consistently deliver high-quality
   products that fulfills customer expectations 
\item monitor and track project status during all steps, to meet deadlines
   and ensure all scope changes,
 variances and contingences that may arise during the 
 projects lifecycle were visible to all people involved
 \item  be a go-to person in the team.
 \item 
   Ability to prioritize and track multiple projects in parallel,
    manage  allocation of resources within the team, supervise and support my team
    activities
\item test plan and test cases creation using TestLodge
\item promote new strategies to speed up effective communication among all
   teams involved in the project,
   improving workflows and defining new internal procedures
\item debugging and monitoring of customized web analytics solutions and 
technical requirements (for example requirements involving
   akamai technologies)
   %(Google
%   Analytics with e-commerce, AdWords, AdSense, IBM coremetrics,
%   SiteCatalysts, I2A, etc) and tag managers (GTM,
%   Tealium) 
%\item training and integration of people who join the company;
   %providing a
%   thoroughly overview of tools, workflows and responsibilities.
   %about 
   \item developing small scripts (mainly using Python+JQL) which help to retrieve project
 information from Jira and provide automatic statistical reports about project status
 \end{itemize} 
   Personally responsible as QA analyst of:
   Camelot Group (UK National Lottery, including the
   launch of the mobile website for the new lotto raffle on October 2013), Dell Inc. (22 countries
   around the world including US, UK, China etc), FedEx,
   MaryKay, Selfridges, Surfstitch and many others.
   Proven ability to work under pressure responsibly and fulfilling high
   expectations.
I have been involved in 247 QA support, which requires ability to quickly gain
familiarity with unknow projects, providing insights which help to spot the root
cause of the issues and testing if it has
been solved successfully.
Company website: \url{http://usablenet.com/}}
%\cventry{July 2015 -- present}{Freelance}{Part-time freelance web developer}{Udine,
%   Italy}{}{
%   Working on small projects to build simple websites using mainly Wordpress.
%   Example of project:
%   \url{http://www.grupoeuroamerica.com.mx}
%}
\cventry{2006 -- 2012}{University of Udine}{Teacher (on call)}{Udine}{}{
   I have been asked to teach  the \emph{Esercitazioni guidate di Fisica per il Corso
      di preparazione al Test di ammissione alla Facolt\`a di Medicina e
      Chirurgia} at the University of Udine (support training lectures
   organized for students who have to perform the examination test to enter the
   first year at the faculty of Medicine).}
\cventry{September~2011 -- August~2012}{R.U.E. Risorse Umane Europa (no-profit
   association)}{IT technical expert (on call)}{Udine}{}{
   Handling the IT issues in a small office (5 employees): pc, network and
   website maintenance}
\cventry{January 2010 -- December 2010}{Consorzio per la Fisica}{\LaTeXe\ typesetter}{Trieste}{}
{   I typeset prof. E.~Gozzi lectures notes for his course of Quantum Mechanics.
   The notes are currently available at 
   \url{http://www-dft.ts.infn.it/~gozzi/QM2.pdf}}
\cventry{February~2008 -- May~2008}{I.N.F.N. (Istituto Nazionale di Fisica
   Nucleare)}{C++ developer as volounteer}{Trieste}
{}{Development of a object-oriented C++ library for uniform and non-uniform
   pseudo-random number generations, including some state-of-art algorithms}

\section{Education and training}
\cventry{September 2009 -- now}%
{Master student in Theoretical Physics}
{University of Trieste}
{Trieste (Italy)}{Not completed yet}
{Not completed yet due to work. I have successfully done  exams including: advanced statistical mechanics,
   field theory, advanced mathematical methods and computational physics,
   including C++
   implementation and comparison of symplectic algorithms for numerical
   integration of ordinary differential initial valued problems in classical
   molecular dynamics and monte carlo simulation of statistical mechanics
   system like Ising model}
%\end{shaded}

\cventry{September~2009}
{Bachelor's degree in Physics}
{University of Trieste}
{Trieste (Italy)}
{110/110}
{Thesis title: \emph{Simulazione numerica dello stress termomeccanico in un
      ellissometro}.
   Advisor: Prof. E.~Milotti.
   Description:
Numerical investigation of the influence of laser beam-pointing fluctuations 
on the
   thermomechanical stress-induced birefringence in the optical
   ellipsometer of PVLAS experiment. The thesis required 1~year of advanced C++ programming
   including the development of an object-oriented library for finite-element
   analysis in rectangular domains and numerical integration of Ito stochastic differential
   equations.
Thesis available (in Italian) at 
\url{http://www.infn.it/thesis/PDF/getfile.php?filename=3304-Candolini-triennale.pdf}
}
    

\cventry{July~2005}
{Scientific high school diploma} 
{Liceo Scientifico Statale G. Marinelli}
{Udine (Italy)}
{100/100}
{High school's thesis in Physics: \emph{Approccio spazio-temporale globale alla
      teoria quantistica e formulazione di Feynman della QED}. Advisor: Prof.
   F.~de~Stefano}
 

\section{Continuing Education}


\cventry{December 17, 2014}
%{SUSY 2013 --- 21st International Conference on
%   Supersymmetry and Unification of Fundamental Interactions}
{Machine Learning}
{Coursera}
{}
{}
{Statement of Accomplishment of Machine Learning course by Professor Andrew Ng
   from Standford University.
   Website of the course:
   \url{https://www.coursera.org/learn/machine-learning/home/info}}


\cventry{20 August 2013 - 23 August 2013}
%{SUSY 2013 --- 21st International Conference on
%   Supersymmetry and Unification of Fundamental Interactions}
{School on Supersymmetry and Unification of Fundamental Interactions (Pre-SUSY
   2013)}
{ITCP}
{Trieste (Italy)}
{}
{Attended the School on Supersymmetry and Unification of
   Fundamental Interactions as partecipant. Website of the school:
   \url{http://presusy2013.ictp.it/}}
\cventry{13 May 2013 - 17 May 2013}
{Workshop on Ultracold Atoms \& Gauge
   Theories}
{ICTP}
{Trieste (Italy)}{}
{Attended Workshop Ultracold Atoms \& Gauge Theories as
   partecipant at the International Center for Theoretical Physics (ICTP).
   Webpage of the course:
   \url{http://cdsagenda5.ictp.trieste.it/full_display.php?ida=a12184}}
\cventry{20 August 2012 -- 24 August 2012}
{Workshop on Majorana Fermions, Non-Abelian Statistics and Topological Quantum Information Processing}%
{ICTP}
{Trieste (Italy)}
{}
{Attended Workshop on Majorana fermions, Non-abelian Statistics and Topological
   Quantum Information Processes as
   partecipant at the International Center for Theoretical Physics (ICTP).
   Webpage of the workshop:
   \url{http://cdsagenda5.ictp.trieste.it/full_display.php?email=0&ida=a11183}}
%\cventry{August 2012}
% arguments 3 to 6 can be left empty
%\cventry{September~2009--now}{}{Institution}{City}{\textit{Grade}}{Description}
%\begin{shaded}

\cventry{February~2010}%
{Lectures of Introduction to Bayesian methods}
{University of Trieste}
{Trieste (Italy)}{}
{I attended \emph{as volountear} the Ph.D. course of Introduction to Bayesian methods (Prof.
   E.~Milotti).
   Lectures addresses Ph.D. students in Physics at university of Trieste.
Topics covered include: Bayesian inference, Maximum-Entropy and its applications
to image restoration, EM algorithm, Markov-Chain Monte Carlo, introduction to
naive Bayesian learning and 
Bayesian classifiers (AUTOCLASS). 
Webpage of the course:
\url{http://wwwusers.ts.infn.it/~milotti/Didattica/Bayes/Bayes.html}
}





\section{Other experiences}
\cventry{2000-2007}{A.F.A.M. (Associazione Friulana di Astronomia e
   Meteorologia}{}{Remanzacco, Udine (Italy)}{}{
   Data reduction of VHF radio forward scatter meteor
   observations and detection of Jupiter decametric radio emissions (as a part
   of NASA's radio JOVE project).
   Data reduction for meteor activity were published monthly on the official
   RMOB bolletin and, at least in one occasion, cited in a referred paper by
   Alastair McBeath: WGN \textbf{31}:264--68 (2003).
   Observations on Quadrantids were published in a referred paper. 
   \newline{}
   Several small projects of statistical data analysis and data visualization in C and C++.
   Created teaching notes about Io-Jupiter radio emissions for the AFAM website. 
   Webpage of the AFAM radioastronomy group is (in Italian) 
   \url{http://www.tng.iac.es/users/boschin/RadioAFAM/}
}

\section{Scholarships}
\cventry{2009}{University College for Sciences ``Luciano Fonda''}{}{Trieste
   (Italy)}{}{
   I won the scholarship in Physics for academic achievements.
   The evaluation of the candidates was based on an oral examination and the documents supporting the
   application (curriculum vit\ae{} et studiourum, certificate indicating the
   exams sat and the marks obtained and two letters of presentation).}
\cventry{2005--2008}{University College for Sciences ``Luciano Fonda''}{}{Trieste
   (Italy)}{}{
   I won the scholarship in Physics for academic achievements and maintaned it
   for the three years of undergraduate studies.
   The evaluation of the candidates was based on writing and oral examination.
   In order to maintain the right to the scholarship students had to sit all
   the exams set for each year within the following 31st October and to obtain
   an \emph{average mark of at least 27/30} for the exams sat in the academic
   year and \emph{no less than 24/30} in any one exam.}

\section{Pubblications}
\cvitem{2003}{
W. Boschin, D. Ganzini, \emph{A. Candolini}, G. Candolini, 
\textbf{Radio Observations of the 2002 December Ursids from North-Eastern
   Italy}, WGN, \textbf{31}:1 29--30 (2003)}

\section{Languages}
\cvitemwithcomment{Primary Language}{Italian}{}
\cvitemwithcomment{Secondary Language}{English (level B2 advanced)}
{2006, University of Cambridge First Certificate (F.C.E.)}

\section{Computer science skills} 
\cvitem{OS}{Linux, MacOS, Windows}
\cvitem{Programming Languages}{Knowledge of C++,  Android, Python, C,  FORTRAN~90.
   Experiences with D, Java, Shell BASH,
   HTML}
\cvitem{Data-interchange format}{JSON, XML}
\cvitem{Database}{SQL, JQL}
\cvitem{IDE and editors}{VI (I'm a proud VI fan), Sublime Text Editor, Eclipse, Android Studio}
\cvitem{Typesetting}{\LaTeXe} 
\cvitem{Tools for software development and testing}{Atlassian Inc. JIRA, Asana,
   Atlassian Confluence, Atlassian 
   TestLodge, CVS, command-line git, GitHub, BitBucket,
   SourceTree, Asana}
\cvitem{Programming environments}{Mathematica,
   Octave/MATLAB}
\cvitem{Scienfic libraries and tools}{Cern ROOT framework, gnuplot, Asymptote Vector Graphics Language, METAPOST,
   GNU Scientific Library, fftw3, HDF5}

\section{Scientific Interests}
\cvlistitem{Quantum and Statistical Field Theory}
\cvlistitem{Statistical data analysis and big data}
\cvlistitem{Object-oriented high-performance scientific computing}
\cvlistitem{D programming language}

\section{Hobbies}
\cvlistitem{Molecular cocktails, modern mixology, bartendering and gin}
\cvlistitem{Listening to blues and progressive rock (expecially Eric
   Clapton, Joe Bonamassa, Mark Knopfler, Eric Johnson, Deep Purple)}
\cvlistitem{Reading books (I'm a proud fan of Dostoevskij)}
\cvlistitem{Typography (I love R. Bringhurst's \emph{Elements of Typographic
      Style}, among my favourite fonts are MinionPro opticals, and I enyoj using \LaTeX{} to implement typographic finesse)}
\cvlistitem{Walking with my Siamese cat Ciambella}

\renewcommand{\listitemsymbol}{-~}            % change the symbol for lists

%\section{Extra 2}
%\cvlistdoubleitem{Item 1}{Item 4}
%\cvlistdoubleitem{Item 2}{Item 5\cite{book1}}
%\cvlistdoubleitem{Item 3}{}

% Publications from a BibTeX file without multibib
%  for numerical labels: \renewcommand{\bibliographyitemlabel}{\@biblabel{\arabic{enumiv}}}
%  to redefine the heading string ("Publications"): \renewcommand{\refname}{Articles}
%\nocite{*}
%\bibliographystyle{plain}
%\bibliography{publications}                   % 'publications' is the name of a BibTeX file

% Publications from a BibTeX file using the multibib package
%\section{Publications}
%\nocitebook{book1,book2}
%\bibliographystylebook{plain}
%\bibliographybook{publications}              % 'publications' is the name of a BibTeX file
%\nocitemisc{misc1,misc2,misc3}
%\bibliographystylemisc{plain}
%\bibliographymisc{publications}              % 'publications' is the name of a BibTeX file

\end{document}
\clearpage
%-----       letter       ---------------------------------------------------------
% recipient data
\recipient{Company Recruitment team}{Company, Inc.\\123 somestreet\\some city}
\date{January 01, 1984}
\opening{Dear Sir or Madam,}
\closing{Yours faithfully,}
\enclosure[Attached]{curriculum vit\ae{}}     % use an optional argument to use a string other than "Enclosure", or redefine \enclname
\makelettertitle

Lorem ipsum dolor sit amet, consectetur adipiscing elit. Duis ullamcorper neque sit amet lectus facilisis sed luctus nisl iaculis. Vivamus at neque arcu, sed tempor quam. Curabitur pharetra tincidunt tincidunt. Morbi volutpat feugiat mauris, quis tempor neque vehicula volutpat. Duis tristique justo vel massa fermentum accumsan. Mauris ante elit, feugiat vestibulum tempor eget, eleifend ac ipsum. Donec scelerisque lobortis ipsum eu vestibulum. Pellentesque vel massa at felis accumsan rhoncus.

Suspendisse commodo, massa eu congue tincidunt, elit mauris pellentesque orci, cursus tempor odio nisl euismod augue. Aliquam adipiscing nibh ut odio sodales et pulvinar tortor laoreet. Mauris a accumsan ligula. Class aptent taciti sociosqu ad litora torquent per conubia nostra, per inceptos himenaeos. Suspendisse vulputate sem vehicula ipsum varius nec tempus dui dapibus. Phasellus et est urna, ut auctor erat. Sed tincidunt odio id odio aliquam mattis. Donec sapien nulla, feugiat eget adipiscing sit amet, lacinia ut dolor. Phasellus tincidunt, leo a fringilla consectetur, felis diam aliquam urna, vitae aliquet lectus orci nec velit. Vivamus dapibus varius blandit.

Duis sit amet magna ante, at sodales diam. Aenean consectetur porta risus et sagittis. Ut interdum, enim varius pellentesque tincidunt, magna libero sodales tortor, ut fermentum nunc metus a ante. Vivamus odio leo, tincidunt eu luctus ut, sollicitudin sit amet metus. Nunc sed orci lectus. Ut sodales magna sed velit volutpat sit amet pulvinar diam venenatis.

Albert Einstein discovered that $e=mc^2$ in 1905.

\[ e=\lim_{n \to \infty} \left(1+\frac{1}{n}\right)^n \]

\makeletterclosing

%\clearpage\end{CJK*}                         % if you are typesetting your resume in Chinese using CJK; the \clearpage is required for fancyhdr to work correctly with CJK, though it kills the page numbering by making \lastpage undefined
\end{document}


%% end of file `template.tex'.
